
\chapter{Using This Template}
The approach to this template is to result in \LaTeX~source files (i.e.,
\texttt{.tex} files) that are as simple as possible.  This is particularly
useful for the first few pages, for example the title page, dedication, and
abstract page, which are difficult to make in \LaTeX~and are supposed to go in
a certain order.

You are welcome to modify \texttt{thesis-umich.cls} to suit your needs or keep
up with modifications to Rackham guidelines. I used this template to submit my
dissertation in 2019, but Rackham can change their rules at any time. If you
have not modified a \texttt{.cls} file before, but know how to define commands
in \TeX, you should not have much trouble, apart from the sneaky \verb=@=
character, which behaves like a letter in \texttt{.cls} files but not in
\texttt{.tex} (see \href{https://tex.stackexchange.com/q/8351/21027}{``What do
  \texttt{\textbackslash makeatletter} and \texttt{\textbackslash makeatother}
do?'' on \TeX\ StackOverflow} and
\href{https://tug.org/pipermail/tugindia/2002-January/000178.html}{``Makeatletter
and Makeatother'' on the \TeX\ Users Group} for more information).

Finally, you should be able to use your preferred bibliography manager (BibTex, BibLaTeX, NatBib, etc.). If the \texttt{\textbackslash bibliography} command is causing trouble, look in \texttt{thesis-umich.cls}, because it was modified there. You can also look at that code for help with formatting bibliographies that are displayed with other commands.
\nocite{*}
\bibliographystyle{apsrev4-2custom}
\bibliography{thesis-bib}
